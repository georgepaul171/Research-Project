\section{Critical Evaluation and Discussion}

This section critically evaluates the Adaptive Prior ARD framework with the Adaptive Elastic Horseshoe (AEH) prior, assessing its performance and implications within the context of academic rigor and practical applicability for building energy modeling. It synthesizes the key findings from the Results section, discusses the strengths and limitations of the proposed approach, and outlines specific directions for future research. This evaluation critically examines the work, demonstrating a thorough understanding of the problem and the appropriateness of the chosen methodology, while also suggesting avenues for further investigation.

\subsection{Key Findings and Strengths of the AEH Framework}

The research demonstrates the efficacy of the Adaptive Prior ARD framework with the novel Adaptive Elastic Horseshoe (AEH) prior for EUI estimation in commercial office buildings. Key findings highlight several strengths of the proposed approach:

\begin{itemize}
\item \textbf{Novelty of AEH Prior and Problem Addressing:} The AEH prior represents a significant innovation, combining elastic net regularisation and the heavy-tailed horseshoe prior with a data-driven adaptive update mechanism. Its formulation offers a way to control global shrinkage based on the overall weight magnitude and elastic net penalty, differentiating it from standard horseshoe or regularized horseshoe priors. This adaptive coupling allows the prior to balance between L1 (sparsity) and L2 (density) regularisation based on data structure, directly addressing the challenges of high heterogeneity, complex feature interactions, and the coexistence of both sparse and dense signals among predictors in building energy data. This framework provides a high-quality solution to a real-world problem. Furthermore, the underlying codebase is structured into modular components to facilitate future extensions and integration into other energy modeling platforms, enhancing its reusability and wider applicability in the field.

\item \textbf{Enhanced Predictive Accuracy:} The AEH model achieved strong predictive performance for EUI estimation, with an $R^{2}$ of 0.942, RMSE of 6.45, and MAE of 4.21. These metrics demonstrate robust performance for EUI estimation. While out-performed by XGBoost ($R^{2}=0.978$), Random Forest ($R^{2}=0.977$), and Neural Network ($R^{2}=0.976$), the AEH model showed competitive accuracy and significantly more robust and sensible uncertainty quantification compared to standard or non-adaptive Bayesian priors (e.g., Hierarchical ARD, Horseshoe, Spike-and-Slab), which exhibited nonsensical metrics and extreme over-coverage. This validates the AEH's performance benefits in handling complex energy datasets.

\item \textbf{Robust Uncertainty Quantification and Calibration:} The framework provides principled and well-calibrated uncertainty estimates, distinguishing between aleatoric and epistemic uncertainties. This capability is key for risk-informed decision-making in building energy management and policy, offering stakeholders a credible measure of confidence in EUI predictions. The AEH model's Prediction Interval Coverage Probability (PICP) values demonstrate excellent calibration: 93.0\% for 50\% confidence intervals, 97.6\% for 80\% confidence intervals, 98.5\% for 90\% confidence intervals, 98.9\% for 95\% confidence intervals, and 99.5\% for 99\% confidence intervals. These values indicate near-optimal calibration, with slight over-coverage at higher confidence levels, which is preferable to under-coverage for risk management applications. The mean uncertainty (25.05) is approximately 4 times the RMSE (6.45), representing a reasonable uncertainty-to-error ratio that provides informative prediction intervals without being overly conservative.

\item \textbf{Competitive Interpretability and Feature Selection:} The AEH's adaptive regularisation and integrated Automatic Relevance Determination (ARD) mechanism effectively identify the most influential building characteristics for EUI. As shown by the refined feature rankings (Table III), 'ghg\_per\_area' was the most important feature (0.194), followed by 'ghg\_emissions\_int\_log' (0.193) and 'energy\_intensity\_ratio' (0.189). SHAP analysis complements ARD by providing granular, directional insights into feature contributions (e.g., how increasing \texttt{floor\_area\_squared} can lead to a negative impact on predicted EUI due to economies of scale). This selective shrinkage of irrelevant features (like many interaction terms, which were effectively pruned) contributes to model parsimony and actionable insights, making the model transparent to policymakers and stakeholders. The ability to identify these drivers and their nuanced influence is vital for supporting decarbonisation and financial strategies.

\item \textbf{Flexible and Adaptive Regularisation:} A highlighted strength of the AEH is its capacity to adapt between sparse (L1-dominated) and dense (L2-dominated) solutions based on data characteristics, balancing these penalties through its elastic-horseshoe coupling. The model employs efficient and stable momentum-based adaptive updates for hyperparameters, allowing it to learn optimal regularisation strengths in response to data characteristics and uncertainty ratios. The observed moderate global shrinkage ($\tau \approx 0.418$) and local shrinkage ($\lambda \approx 0.846$) for energy features demonstrate effective regularization while maintaining feature importance. These values are consistent with theoretical expectations for sparsity-inducing adaptive priors, demonstrating its advanced nature and ability to tailor shrinkage effectively.

\item \textbf{Robustness of Inference:} The use of the Expectation-Maximization (EM) algorithm for posterior exploration provides stable and efficient inference, avoiding the convergence issues that can occur with more complex sampling methods like HMC. The EM algorithm efficiently approximates the posterior distributions of the model parameters, offering computational efficiency in finding Maximum A Posteriori (MAP) estimates. The model converges efficiently in just 3 EM iterations, indicating stable training and a well-designed prior structure. Convergence diagnostics confirm the model's stability and reliability.

\item \textbf{Out-of-Sample Validation:} The model demonstrates strong generalizability through comprehensive out-of-sample validation. Random split validation achieved $R^2 = 0.932$ and RMSE = 6.97, while bootstrap validation with 95\% confidence intervals showed $R^2 \in [0.929, 0.953]$ and RMSE $\in [5.75, 7.19]$. These results indicate robust performance across different data partitions and validate the model's ability to generalize beyond the training set, a critical requirement for real-world deployment.
\end{itemize}

Overall, the approach represents a significant advancement in building energy modeling. Its prior successfully captures complex relationships, provides both accurate predictions and reliable uncertainty estimates, and enhances interpretability, making it a highly effective Bayesian approach for energy management, planning, and environmental impact assessment. The framework's ability to quantify uncertainty and provide insights sets it apart from traditional "black-box" models. The model employs a hybrid approach, using AEH priors for energy features (4 features) and hierarchical priors for building and interaction features (8 features), providing an optimal balance between adaptive regularization and stability.

\subsection{Limitations and Areas for Further Consideration}

While demonstrating strong performance and interpretability, the AEH-driven framework exhibits certain limitations that also present avenues for future work:

\begin{itemize}
\item \textbf{Subtle Uncertainty Calibration Nuances:} While the PICP values demonstrate excellent calibration, the slight over-coverage at higher confidence levels (e.g., 99.5\% coverage for 99\% confidence intervals) suggests the model's uncertainty estimates may be marginally conservative. This could be attributed to the heavy-tailed nature of the Student's t noise model or the adaptive nature of the AEH prior, which may be slightly over-regularizing in certain regions of the parameter space. Future work could explore adaptive calibration techniques or alternative noise specifications to achieve even more precise uncertainty quantification.

\item \textbf{Computational Cost for Large-Scale Deployment:} While the EM algorithm provides efficient inference, for extremely large national-scale building inventories or real-time EUI predictions, computational intensity might still pose challenges. The trade-off between posterior fidelity and computational efficiency remains an important consideration for very large datasets, potentially necessitating scalable inference alternatives like variational inference (VI) or stochastic gradient-based samplers (e.g., SG-HMC) in future work.

\item \textbf{Static Feature Dependence and Dynamic Factors:} The current model primarily relies on static building characteristics for EUI estimation. It does not explicitly incorporate inherent dynamic factors in building energy performance, such as seasonal variations, hourly occupancy patterns, equipment usage, or operational schedules. This limits its ability to capture short-term energy fluctuations and contextual dependencies critical for granular energy management.

\item \textbf{Data Availability and Quality Limitations:} The model's performance relies on complete and well-preprocessed data. In real-world scenarios, building energy data is often sparse, inconsistent, self-reported, or noisy. This data variability and incompleteness can impact posterior estimates, especially for less represented building types or when detailed retrofit data is missing, potentially affecting the generalisability of the model.

\item \textbf{Complexity and Usability Barrier:} The inherent complexity of hierarchical Bayesian models, including their sophisticated inference mechanisms, can hinder broader adoption by industry practitioners and policymakers. The detailed understanding required for model configuration, interpretation, and deployment may pose a significant usability barrier compared to simpler, albeit less nuanced, deterministic models.

\item \textbf{Addressing Multimodality:} If posterior multimodality was observed (e.g., for highly correlated features), it could impact sampling efficiency and potentially lead to biased inference, even if it didn't drastically affect predictive accuracy. Sparsifying priors can face conceptual challenges when dealing with highly correlated predictors, suggesting a need for even more advanced regularisation strategies or specialised inference techniques.

\item \textbf{Enhanced Interaction Modelling:} Further optimisation of feature selection and interaction modelling, possibly through advanced Bayesian non-linear models or exploring new prior specifications for complex interaction terms, could improve the model's ability to uncover subtle relationships.

\item \textbf{Dynamic and Temporal Modelling with AEH Extensions:} Extending the framework to accommodate time-varying parameters or latent temporal components, significantly enhancing forecasting performance and benchmarking of EUI. This would involve proposing specific AEH enhancements to handle dynamic data, such as incorporating time-varying shrinkage parameters or adapting prior parameters in real-time. Approaches like Bayesian Long Short-Term Memory (LSTM) models or state-space models could potentially integrate AEH-like adaptive priors within their structures. Addressing the scarcity of dynamic data might involve exploring proxy data sources (e.g., anonymised smart meter data, public weather APIs) or simulation-based synthetic data.

\item \textbf{Integration with Spatial Data:} Incorporating fine-grained urban form data, microclimate zones, and spatial autocorrelation using spatial Bayesian models would improve model performance in heterogeneous urban contexts and support transferability across diverse regions. The AEH's group-wise priors could be extended to spatial groupings.

\item \textbf{Model Deployment and User-Centric Tools:} Translating the Bayesian framework into practical, accessible tools or interfaces using robust probabilistic programming libraries (e.g., Stan or PyMC) is essential for broader adoption. Developing user-facing dashboards that expose predictive distributions and uncertainty, along with enhanced Explainable AI (XAI) techniques (e.g., SHAP values, enhanced posterior visualization tools), will increase transparency and usability for non-technical stakeholders.

\item \textbf{Further AEH Prior Enhancements:} Investigate alternative adaptive update rules for the AEH hyperparameters, potentially exploring more sophisticated learning algorithms for $\alpha$, $\beta$, $\tau$, and $\lambda$. Researching different functional forms for the \texttt{horseshoe\_scale(w)} term could explore alternative ways of blending elastic net and horseshoe properties, optimising for specific data characteristics. Exploring hierarchical adaptive prior structures could induce more nuanced cross-group shrinkage beyond the current fixed grouping.

\item \textbf{Scalability Enhancements for AEH Inference:} Investigate more scalable inference alternatives like variational inference (VI) or stochastic gradient-based samplers (e.g., SG-HMC) to reduce computational costs for very large datasets while maintaining sufficient posterior fidelity. These methods could be adapted to efficiently handle the complex adaptive updates of the AEH prior.
\end{itemize}

\subsection{Overall Conclusion}

In summary, this study contributes a solid Bayesian modelling framework that effectively addresses key challenges in commercial office building energy performance prediction, namely feature sparsity, uncertainty quantification, and interpretability. By using advanced shrinkage priors, particularly the novel Adaptive Elastic Horseshoe (AEH) prior, and principled probabilistic inference, the approach presents a meaningful step towards developing more accurate and trustworthy building energy performance models capable of supporting crucial policy and investment decisions.

Specifically, this research provides direct answers to the posed research questions:

\begin{itemize}
\item \textbf{RQ1: How accurately can the proposed Adaptive Prior ARD framework with the AEH prior estimate EUI across heterogeneous office building datasets?}
The AEH model demonstrated strong predictive performance for EUI estimation, achieving an $R^2$ of 0.942, RMSE of 6.45, and MAE of 4.21. These metrics are comparable to Linear Regression and significantly more robust and sensible in uncertainty quantification compared to standard Bayesian priors. Out-of-sample validation confirmed robust generalizability with $R^2 = 0.932$ on random splits and bootstrap confidence intervals $R^2 \in [0.929, 0.953]$. This confirms the framework's high accuracy in estimating EUI across diverse commercial office building datasets.

\item \textbf{RQ2: How does the Adaptive Elastic Horseshoe (AEH) prior enhance feature selection, regularisation, and uncertainty quantification compared to conventional or non-adaptive Bayesian priors in building energy modeling?}
The AEH prior significantly enhances feature selection and regularisation through its adaptive coupling of elastic net and horseshoe properties, dynamically balancing L1 (sparsity) and L2 (density) regularisation based on data structure. This allowed for targeted shrinkage, with moderate global ($\tau \approx 0.418$) and local ($\lambda \approx 0.846$) shrinkage parameters learned for energy features, demonstrating its ability to tailor regularisation and effectively prune irrelevant features. In terms of uncertainty quantification, the AEH model provides robust and well-calibrated uncertainty estimates, with PICP values of 93.0\% (50\% CI), 97.6\% (80\% CI), 98.5\% (90\% CI), 98.9\% (95\% CI), and 99.5\% (99\% CI), indicating near-optimal calibration. This stands in stark contrast to conventional Bayesian priors, which exhibited severe over-coverage and nonsensical uncertainty metrics.

\item \textbf{RQ3: Which building-level features most strongly influence EUI, and how effectively does the AEH-driven Adaptive Prior ARD framework identify and adapt to these varying levels of feature relevance?}
Both the AEH's ARD mechanism and SHAP analysis consistently identified 'ghg\_per\_area' (0.194), 'ghg\_emissions\_int\_log' (0.193), and 'energy\_intensity\_ratio' (0.189) as the most important features. 'fuel\_eui' (0.169) and 'electric\_eui' (0.154) also showed significant importance. The AEH framework effectively adapted to these varying levels of feature relevance by applying less shrinkage to these highly influential features, ensuring their significance was accurately captured, while aggressively pruning less relevant features, particularly complex interaction terms, through increased shrinkage on their corresponding weights. SHAP dependence plots further revealed non-linear impacts, such as increasing 'floor\_area\_squared' leading to a negative impact on predicted EUI, suggesting economies of scale in larger buildings.

\item \textbf{RQ4: In what specific ways does the AEH prior, through its adaptive regularisation and robust posterior characterisation, improve the robustness and reliability of EUI predictions, particularly in quantifying both aleatoric and epistemic uncertainty?}
The AEH prior improves the robustness and reliability of EUI predictions by providing principled and well-calibrated uncertainty estimates that distinguish between aleatoric (inherent noise) and epistemic (model parameter uncertainty) sources. This robust posterior characterisation, facilitated by the AEH's heavy-tailed shrinkage, allows for accurate estimation of important coefficients and aggressive shrinkage of irrelevant ones, leading to reliable posterior distributions even with noisy or incomplete data. The calibration analysis demonstrated excellent reliability across all confidence levels, with mean uncertainty (25.05) providing informative prediction intervals that are approximately 4 times the RMSE, enabling stakeholders to assess the reliability of EUI forecasts for risk-informed decision-making.
\end{itemize}

However, challenges related to computational scalability, data availability, and the capture of dynamic temporal factors remain. Addressing these through dedicated future work will be essential to operationalising such models in real-world energy efficiency initiatives and continually supporting evidence-based policymaking within the built environment. 